\documentclass[11pt,a4paper]{article}

% ---------- PACKAGES ----------
\usepackage{graphicx}
\usepackage{amsmath}
\usepackage{geometry}
\usepackage{setspace}
\usepackage{titlesec}
\usepackage{float}
\usepackage{caption}

% ---------- PAGE SETUP ----------
\geometry{margin=1in}
\setstretch{1.2}
\titleformat{\section}{\large\bfseries}{\thesection}{1em}{}

\begin{document}

% ---------- LOGO (HIGHER UP, TOP-RIGHT) ----------
\begin{flushright}
    \vspace*{-1cm} % pushes logo close to top edge
    \includegraphics[width=3cm]{logo.png} % <-- ensure this file exists in same folder
\end{flushright}

% ---------- CENTERED TITLE BLOCK ----------
\vspace{1cm} % space between logo and title block
\begin{center}
    \textbf{\Large Data Mining – Exercise Sheet 2}\\[0.5em]
    \LARGE Simulation of Randomised Rumor Spreading Protocols\\[1em]

    \normalsize
    \textbf{Group Members:}\\
    Mohammad Umar\\
    Yugal Verma\\
    Adam Bielecki\\[0.5em]

    Hamburg University of Technology\\
    \texttt{(Group Project Submission)}\\[1em]
    \today
\end{center}

\vspace{1cm}

% ---------- CONTENT ----------
\section{Introduction}
The objective of this project is to simulate and compare three rumor spreading protocols—\textbf{Push}, \textbf{Pull}, and \textbf{Push–Pull}. These protocols model how information propagates through a network of agents. Each agent can either share or request information from another random agent in successive rounds until the entire network is informed.

This exercise builds upon our previous work, where we used Python for data visualization and statistical analysis. The experience gained in using \textbf{Seaborn} for advanced visualization in the last project was particularly helpful for designing cleaner and more interpretable comparative plots in this assignment.

\section{Methodology}
The simulation was implemented in Python using the \texttt{random} and \texttt{matplotlib.pyplot} modules.

\subsection*{Push Model}
In the Push model, each informed agent randomly selects another agent in every round and shares the rumor. The process repeats until all agents become informed.

\subsection*{Pull Model}
In the Pull model, uninformed agents randomly contact another agent in each round. If the contacted agent is informed, the rumor is pulled successfully.

\subsection*{Push–Pull Model}
In the combined Push–Pull model, both interactions occur simultaneously: informed agents share information, and uninformed agents attempt to receive it. This generally results in faster information dissemination.

\section{Implementation}
The implementation consists of three main simulation functions—\texttt{simulate\_push}, \texttt{simulate\_pull}, and \texttt{simulate\_push\_pull}. Each function returns the number of rounds required to inform all agents for a given population size $n$.

The simulation campaign tested populations of $n = 10, 50, 100, 500, 1000,$ and $5000$ agents. The results were recorded for each protocol and plotted using \texttt{matplotlib} and \texttt{seaborn}.

\section{Results and Discussion}
The figures below compare the number of rounds required to spread the rumor under different protocols and visualize the spread intensity across multiple simulations.

\begin{figure}[H]
    \centering
    % Each image takes up roughly half the page width
    \begin{minipage}[t]{0.48\textwidth}
        \centering
        \includegraphics[width=\linewidth]{lineplot.png}
    \end{minipage}
    \hfill
    \begin{minipage}[t]{0.48\textwidth}
        \centering
        \includegraphics[width=\linewidth]{heatmap.png}
    \end{minipage}
    \caption{Comparison of rumor spreading results: the \textbf{Line Plot} (left) shows convergence speed of Push, Pull, and Push–Pull models, while the \textbf{Heatmap} (right) visualizes spread intensity and efficiency across varying population sizes.}
\end{figure}

The line plot demonstrates that the Push–Pull model outperforms the other two, achieving faster convergence in fewer rounds. The heatmap helps visualize the relative efficiency and density of information flow, reinforcing the hybrid model's advantage in faster propagation.

\section{Reflection and Learning}
From a computational perspective, this simulation reinforced our understanding of iterative random processes, convergence, and stochastic modeling. Prior experience using \textbf{Seaborn} in earlier exercises helped us visualize statistical relationships effectively. Although Matplotlib handled the basic visualization, Seaborn’s interface was essential for generating the heatmap and enhancing interpretability.

\section{Conclusion}
This project successfully simulated and compared three rumor spreading strategies. The Push–Pull model consistently achieved the fastest information spread. Future work could explore the effect of probabilistic interactions or constrained network topologies.

\vspace{1em}
\noindent\textbf{Keywords:} rumor spreading, push-pull model, Python simulation, network propagation, data visualization

\end{document}
