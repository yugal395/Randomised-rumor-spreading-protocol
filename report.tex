\documentclass[a4paper,12pt]{article}

% ---------- PACKAGES ----------
\usepackage[utf8]{inputenc}
\usepackage{graphicx}
\usepackage{caption}
\usepackage{float}
\usepackage{geometry}
\usepackage{hyperref}
\geometry{margin=1in}
\hypersetup{colorlinks=true, linkcolor=blue, urlcolor=blue}

% ---------- TITLE ----------
\title{\textbf{Simulation and Visualization of Randomized Rumor Spreading Protocols}}
\author{Yugal Verma \\ Technical University of Hamburg}
\date{\today}

\begin{document}
\maketitle

% ---------- INTRODUCTION ----------
\section{Introduction}
This report presents a simulation study of a simple \textit{randomized rumor spreading protocol}.
In the model, a system of \(n\) agents exchange information in synchronized rounds.
Initially, only one agent is \textit{informed}, and the goal is to spread the rumor to all others.
We analyze three basic protocols: \textbf{Push}, \textbf{Pull}, and \textbf{Push–Pull}.

% ---------- METHODOLOGY ----------
\section{Simulation Methodology}
The simulation was implemented in Python using random interactions between agents:
\begin{itemize}
  \item \textbf{Push:} Informed agents call a random peer to share the rumor.
  \item \textbf{Pull:} Uninformed agents call a random peer and learn the rumor if that peer is informed.
  \item \textbf{Push–Pull:} Both agents exchange information, so if either knows the rumor, both become informed.
\end{itemize}
The program measures the number of rounds required for all \(n\) agents to become informed.
Simulations were performed for \(n=\{10,50,100,500,1000\}\).

% ---------- RESULTS ----------
\section{Results and Discussion}
Figure~\ref{fig:lineplot} shows how the number of rounds increases as the system grows.
The Push–Pull protocol consistently reaches full dissemination in the fewest rounds,
while Push and Pull exhibit similar but slower performance.

% ---------- FIGURE 1 ----------
\begin{figure}[H]
  \centering
  \includegraphics[width=0.8\textwidth]{rumor_spreading_comparison.png}
  \caption{Rounds to spread the rumor for varying numbers of agents across Push, Pull, and Push–Pull protocols.}
  \label{fig:lineplot}
\end{figure}

To illustrate the relative differences across all scales, a heatmap visualization is provided in Figure~\ref{fig:heatmap}.
Lighter colors correspond to faster spreading (fewer rounds).

% ---------- FIGURE 2 ----------
\begin{figure}[H]
  \centering
  \includegraphics[width=0.75\textwidth]{heatmap_comparison.png}
  \caption{Heatmap showing rounds required by each protocol for different agent counts.}
  \label{fig:heatmap}
\end{figure}

% ---------- CONCLUSION ----------
\section{Conclusion}
The Push–Pull protocol demonstrates the highest efficiency, achieving full rumor spread in the fewest rounds.
The number of rounds grows sub-linearly with \(n\), indicating that rumor spreading scales well even in large systems.
Such protocols model efficient data dissemination methods used in distributed systems like Amazon DynamoDB or peer-to-peer networks.

\end{document}
